% !TEX encoding = UTF-8
% !TEX program = lualatex
\documentclass[12pt,b5paper]{ltjsarticle}

% 数学系パッケージ
\usepackage{amsmath, amssymb, amsthm}
\usepackage{graphicx}
\usepackage{hyperref}

% 定理環境の設定
\newtheorem{theorem}{定理}[section]
\newtheorem{lemma}[theorem]{補題}

\title{修士論文用 LaTeX 最小サンプル}
\author{立命館 太郎}
\date{\today}

\begin{document}

\maketitle

\section{はじめに}
本ファイルは LuaLaTeX でコンパイルすることを想定した最小構成のサンプルです。
数式や図の参照方法の確認に使用してください。

\section{数式の例}
インライン数式は $E=mc^2$ のように記述します。
番号付きの別行立て数式は \texttt{align} 環境を使用します。

\begin{align}
    \int_{0}^{\infty} e^{-x^2} dx &= \frac{\sqrt{\pi}}{2} \label{eq:gauss} \\
    \sum_{n=1}^{\infty} \frac{1}{n^s} &= \zeta(s)
\end{align}

式\eqref{eq:gauss}はガウス積分として知られています。

\section{図の例}
図を挿入する場合は以下のように記述します(実際にコンパイルするには画像ファイルが必要です)。

\begin{figure}[htbp]
    \centering
    % \includegraphics[width=0.5\textwidth]{example-image.pdf}
    \caption{図のサンプル(コメントアウトを外して使用)}
    \label{fig:sample}
\end{figure}

図\ref{fig:sample}は、図の配置例を示しています。

\begin{theorem}[ピタゴラスの定理]
直角三角形の斜辺の長さを $c$ とし、他の2辺の長さを $a, b$ とすると、
$a^2 + b^2 = c^2$ が成り立つ。
\end{theorem}

\end{document}
