% !TEX program = lualatex
\documentclass[11pt,professionalfonts]{beamer}

% テーマ・色など(シンプルで無難なものを選択)
\usetheme{Madrid}
\usecolortheme{default}

% 日本語フォント設定(環境に応じて必要なら有効化してください)
% \setmainjfont{Noto Serif CJK JP}
% \setsansjfont{Noto Sans CJK JP}

% 数学パッケージ
\usepackage{amsmath,amssymb,amsthm}

% 図表
\usepackage{graphicx}
\usepackage{booktabs}

% 画像パス(必要に応じて変更)
\graphicspath{{../figures/}}

% タイトル情報(適宜書き換えてください)
\title[修士論文発表タイトル(略称)]{修士論文発表タイトル(正式版)}
\author[氏名]{氏名}
\institute[所属略称]{所属(研究科・専攻・研究室など)}
\date[202X年X月X日]{202X年X月X日 修士論文発表}

\begin{document}

% タイトルスライド
\begin{frame}
  \titlepage
\end{frame}

% 目次(必要に応じて)
\begin{frame}{発表の構成}
  \tableofcontents
\end{frame}

%----------------------------------------
\section{研究の背景と目的}
%----------------------------------------

\begin{frame}{研究の背景}
  \begin{itemize}
    \item 本研究が対象とする問題領域の概要
    \item なぜこの問題が重要なのか(応用先・理論的意義など)
    \item 現状の課題(何がうまくいっていないか)
  \end{itemize}
\end{frame}

\begin{frame}{本研究の目的}
  \begin{itemize}
    \item 解決したい具体的な課題
    \item 本研究で目指すゴール
    \item ゴールを達成するための基本的なアプローチ
  \end{itemize}
\end{frame}

%----------------------------------------
\section{関連研究と位置づけ}
%----------------------------------------

\begin{frame}{関連研究の概要}
  \begin{itemize}
    \item 代表的な既存手法 A
    \item 代表的な既存手法 B
    \item それぞれの長所・短所
  \end{itemize}
\end{frame}

\begin{frame}{本研究の位置づけ}
  \begin{itemize}
    \item 既存研究と比較したときの本研究の立ち位置
    \item 本研究が埋めようとしているギャップ
    \item 想定する利用場面・貢献のイメージ
  \end{itemize}
\end{frame}

%----------------------------------------
\section{提案手法 / 理論的結果}
%----------------------------------------

\begin{frame}{問題設定}
  \begin{itemize}
    \item 扱う対象(データ、空間、確率分布など)
    \item 変数や記号の定義(必要最小限)
    \item 最終的に求めたいもの
  \end{itemize}
  % 数式の例(実際の内容に合わせて書き換えてください)
  \[
    \min_{x \in \mathbb{R}^n} f(x)
  \]
\end{frame}

\begin{frame}{提案手法の概要}
  \begin{itemize}
    \item 提案手法のアイデア(直感的な説明)
    \item 手法の流れ(アルゴリズムのステップなど)
    \item 既存手法との違い・改良点
  \end{itemize}
\end{frame}

\begin{frame}{主な理論的結果}
  \begin{itemize}
    \item 主要な定理または命題
    \item その定理が保証する性質(収束性、バイアスの減少など)
    \item 証明のアイデア(詳細は論文参照)
  \end{itemize}
  % theorem 環境を使う場合(フォントサイズに注意)
  % \begin{theorem}
  %   ここに主要な結果を記述する。
  % \end{theorem}
\end{frame}

%----------------------------------------
\section{実験設定と結果}
%----------------------------------------

\begin{frame}{実験設定}
  \begin{itemize}
    \item 使用したデータセットやシミュレーション条件
    \item 比較対象とした既存手法
    \item 評価指標と実験の目的
  \end{itemize}
\end{frame}

\begin{frame}{実験結果の例}
  \begin{itemize}
    \item 代表的な図・表の説明(詳細は論文参照)
    \item 提案手法が優れている点 / 劣っている点
    \item 結果から得られる考察
  \end{itemize}

  % 図の例(実際のファイル名に置き換えてください)
  % \begin{figure}
  %   \centering
  %   \includegraphics[width=0.7\linewidth]{example-figure}
  %   \caption{実験結果の一例}
  % \end{figure}
\end{frame}

%----------------------------------------
\section{まとめと今後の課題}
%----------------------------------------

\begin{frame}{本研究のまとめ}
  \begin{itemize}
    \item 本研究で扱った問題と目的
    \item 提案手法/結果の要約
    \item 本研究の貢献(理論・応用の両面)
  \end{itemize}
\end{frame}

\begin{frame}{今後の課題}
  \begin{itemize}
    \item 本研究の限界(前提条件、スケーラビリティなど)
    \item 今後検討したい拡張・改善案
    \item 将来的な応用の可能性
  \end{itemize}
\end{frame}

%----------------------------------------
\section*{謝辞}
%----------------------------------------

\begin{frame}{謝辞}
  \begin{itemize}
    \item 指導教員・共同研究者への謝意
    \item データ提供・助成金などへの謝意
  \end{itemize}
\end{frame}

%----------------------------------------
\section*{質疑応答}
%----------------------------------------

\begin{frame}{ご清聴ありがとうございました}
  \centering
  ご清聴ありがとうございました。\\
  ご質問をお受けします。
\end{frame}

% 付録スライド(必要に応じて追加)
% \appendix
% \section*{付録}
% \begin{frame}{補足スライドの例}
%   \begin{itemize}
%     \item 詳細な実験結果
%     \item 追加の理論的結果
%   \end{itemize}
% \end{frame}

\end{document}
